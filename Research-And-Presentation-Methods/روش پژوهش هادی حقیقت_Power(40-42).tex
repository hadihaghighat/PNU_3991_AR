\documentclass[a4,9pt]{beamer}
\usetheme{Frankfurt}
\usepackage{multicol}
\usepackage{xcolor}
\usepackage{graphicx}
\linespread{1.35}
\usepackage{amsmath}
\usepackage{color}
\usepackage{tikz}
\usetikzlibrary{arrows,automata}

\begin{document}

\begin{frame}
\section*{ENSURING QUALITY OF THE LITERATURE REVIEWED}
\begin{flushleft}
\textbf{40}\hspace*{1cm} \texttt{CHAPTER FOUR}
\end{flushleft}

\vspace*{0.5cm}

completed study that answers many of his or her research questions, a careful reading of the work almost always reveals hints and suggestions for ways in which fur-ther research is both necessary and important; thus, such studies can result in impor-tant and beneficial shifts and modifications to the research proposal.\\
\hspace*{0.5cm} In traditional research reports and thesis designs, the literature review is placed near the beginning of the document. This should not be taken as a sign that the liter-ature review is completed before other parts of the study commence. In our experience, creating, fine-tuning, and editing the literature review is an iterative and ongoing process. As the research proceeds, new problems and questions arise that require inves-tigation and input from others--documentation of these questions and their resolution in the literature review enhances the value of the e-research project.\\
\end{frame}

\begin{frame}
\hspace{0.5cm} A well-done literature review will not just document the results of earlier stud-ies, Rather, it will reflect on all aspects of the research process. Further, an excellent lit-erature review has specific characteristics that separate it from a good literature review.\\
the mention of\\

\vspace*{0.2cm}
\begin{itemize}
  \item the theory that guides the research and helps to frame the research question;\\
  \item the methodology used, including the development of techniques and tools used for analyzing and interpreting the results; and\\
  \item the means by which the results are disseminated.\\
\end{itemize}

\vspace*{0.3cm}
\end{frame}

\begin{frame}
\hspace*{0.5cm} As a product, the literature review will serve both you and subsequent researchers as a record of and a set of pointers to the research that you have extracted from the large base of possible knowledge. It represents your informed extraction and synthesis of the extant research and thus is itself a valuable contribution. The literature review will also guide future researchers in understanding why you made the research choices that you did, help others to uncover and recreate the research process, and disclose the literature that you found of greatest value in your research efforts.\\

\vspace*{0.5cm}
\end{frame}

\begin{frame}
\large{
\textbf{ENSURING QUALITY OF THE LITERATURE REVIEWED}
}
\vspace*{0.2cm}

\small{There are five basic elements that academic researchers require of information sources (Kibirige \& Depalo,2000). These are accessibility, timeliness, readability, relevance, and authority. The use of a network changes our approach to, and means of, assessing these quality indicators, but does not change their importance. Nor does the use of a network change the need to access the literature (e.g., books, microfiche archives, and paper-based academic journals) from established libraries at universities or other edu-cational institutions.\\
}

\vspace*{0.3cm}
\large{
\textbf{Accessibility}
}
\vspace*{0.2cm}

\small{The most dramatic impact of a network on the process of building a literature review is the increase in accessibility. For must research, the locale for creating a literature }

\end{frame}

\begin{frame}
\section*{ENSURING QUALITY OF THE LITERATURE REVIEWED}
\begin{flushright}
 \texttt{THE LITERATURE REVIEW PROCESS IN E-RESEARCH} \hspace*{1cm} \textbf{41}
\end{flushright}

\vspace*{0.5cm}
review has shifted from the stacks of the research library to the networked computer screen located in the home, office, or library. Until quit recently, a review of the for-mal published literature was a two-step process. The first step involved searching databases of published abstracts followed by the second step of retrieving the article from the paper or microfiche archives of academic journals, conference proceedings, and reference books. The Net has steadily eroded the time and effort required to undertake both of these by providing direct access to indexes, research data-bases, and to the full text of an increasingly large number of articles, reports, and scholarly books.\\
\hspace*{0.5cm} Accessibility is also increasing as a result of the surge of interest and product availability for wireless products. Although we are not swept up in the wave of hyper-bole that paints a picture of e-researchers completing their literature review while relaxing on a beach, jogging on a sidewalk, or driving to work, we do acknowledge that the amount of valuable research information available ''anywhere/anytime'' continues to grow. Networked information also increases accessibility to resources that would require considerable travel time and effort. The use of Net-based video and audio telephones will allow researchers to meet with, share, and disseminate results with related researchers located anywhere that is ''Net accessible.'' Accessibility for visually handicapped researchers also has increased with the dissemination of research results that conform to standards designed to improve Web page readability for handicapped users (Center for Applied Special Technology,2001).\\

\vspace*{0.5cm} 
\end{frame}
\end{document} 