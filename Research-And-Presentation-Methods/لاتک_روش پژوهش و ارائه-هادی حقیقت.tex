\documentclass{book}
\usepackage{multicol}
\usepackage{xcolor}
\usepackage{graphicx}
\linespread{1.35}
\usepackage{amsmath}
\usepackage{color}
\usepackage{tikz}
\usetikzlibrary{arrows,automata}

\begin{document}

\begin{flushleft}
\textbf{40}\hspace*{1cm} \texttt{CHAPTER FOUR}
\end{flushleft}

\vspace*{0.5cm}

completed study that answers many of his or her research questions, a careful reading of the work almost always reveals hints and suggestions for ways in which fur-ther research is both necessary and important; thus, such studies can result in impor-tant and beneficial shifts and modifications to the research proposal.\\
\hspace*{0.5cm} In traditional research reports and thesis designs, the literature review is placed near the beginning of the document. This should not be taken as a sign that the liter-ature review is completed before other parts of the study commence. In our experience, creating, fine-tuning, and editing the literature review is an iterative and ongoing process. As the research proceeds, new problems and questions arise that require inves-tigation and input from others--documentation of these questions and their resolution in the literature review enhances the value of the e-research project.\\ 

\hspace{0.5cm} A well-done literature review will not just document the results of earlier stud-ies, Rather, it will reflect on all aspects of the research process. Further, an excellent lit-erature review has specific characteristics that separate it from a good literature review.\\
the mention of\\

\vspace*{0.2cm}
\begin{itemize}
  \item the theory that guides the research and helps to frame the research question;\\
  \item the methodology used, including the development of techniques and tools used for analyzing and interpreting the results; and\\
  \item the means by which the results are disseminated.\\
\end{itemize}

\vspace*{0.3cm}

\hspace*{0.5cm} As a product, the literature review will serve both you and subsequent researchers as a record of and a set of pointers to the research that you have extracted from the large base of possible knowledge. It represents your informed extraction and synthesis of the extant research and thus is itself a valuable contribution. The literature review will also guide future researchers in understanding why you made the research choices that you did, help others to uncover and recreate the research process, and disclose the literature that you found of greatest value in your research efforts.\\

\vspace*{0.5cm}

\large{
\textbf{ENSURING QUALITY OF THE LITERATURE REVIEWED}
}
\vspace*{0.2cm}

\small{There are five basic elements that academic researchers require of information sources (Kibirige \& Depalo,2000). These are accessibility, timeliness, readability, relevance, and authority. The use of a network changes our approach to, and means of, assessing these quality indicators, but does not change their importance. Nor does the use of a network change the need to access the literature (e.g., books, microfiche archives, and paper-based academic journals) from established libraries at universities or other edu-cational institutions.\\
}

\vspace*{0.3cm}
\large{
\textbf{Accessibility}
}
\vspace*{0.2cm}

\small{The most dramatic impact of a network on the process of building a literature review is the increase in accessibility. For must research, the locale for creating a literature }

\newpage

\begin{flushright}
 \texttt{THE LITERATURE REVIEW PROCESS IN E-RESEARCH} \hspace*{1cm} \textbf{41}
\end{flushright}

\vspace*{0.5cm}
review has shifted from the stacks of the research library to the networked computer screen located in the home, office, or library. Until quit recently, a review of the for-mal published literature was a two-step process. The first step involved searching databases of published abstracts followed by the second step of retrieving the article from the paper or microfiche archives of academic journals, conference proceedings, and reference books. The Net has steadily eroded the time and effort required to undertake both of these by providing direct access to indexes, research data-bases, and to the full text of an increasingly large number of articles, reports, and scholarly books.\\
\hspace*{0.5cm} Accessibility is also increasing as a result of the surge of interest and product availability for wireless products. Although we are not swept up in the wave of hyper-bole that paints a picture of e-researchers completing their literature review while relaxing on a beach, jogging on a sidewalk, or driving to work, we do acknowledge that the amount of valuable research information available ''anywhere/anytime'' continues to grow.


 Networked information also increases accessibility to resources that would require considerable travel time and effort. The use of Net-based video and audio telephones will allow researchers to meet with, share, and disseminate results with related researchers located anywhere that is ''Net accessible.'' Accessibility for visually handicapped researchers also has increased with the dissemination of research results that conform to standards designed to improve Web page readability for handicapped users (Center for Applied Special Technology,2001).\\

\vspace*{0.5cm}
\large{
\textbf{Timeliness}\\
}

\hspace*{0.7cm}Increases in timeliness are both the bane and triumph of Internet-based search results. The ease of publishing preliminary findings, unreviewed drafts, and final documents on the Web, results in a proliferation of documentation, much of which is by its nature temporary and transitory.

\small{
An older paradigm of research literature saw publication of only those materials that survived a rigorous peer-review process. This time-consuming and lengthy process has been blamed for the creation of bottlenecks that slow dissem-ination and replication of important research results. The most significant time losses occur in the exchange of articles between editors and reviewers. The use of email, cou-pled with the capacity for reviews to comment on and edit electronically has already resulted in improvements to this problem. More radical concepts of review such as the provision for public participation in, and public review of, the peer review process are being developed by a variety of peer-reviewed educational publications (see the \emph{four-nal of Interactive Media in Education } at http://www-jime.open.ac.uk/). The use of the Net by authors to publish their own work (often referred to as the ''vanity press'') is a challenge to authenticity--but a boon to timeliness. Like many commentators (such as Harnad, 1996), we see the Net as not eliminating, but improving the speed and efficacy of the peer-review process.\\
\hspace*{0.5cm} Net resources are plagued with an expectation of currency that far surpasses that expected of paper publications. Since it is possible to update information Web sites regularly, we have come to expect such continuous revision. However, many authors do not take the time and effort to maintain sites once created, and, thus, like a paper\\ }

\newpage
\begin{flushleft}
\textbf{42} \hspace*{1cm} \texttt{CHAPTER FOUR}
\end{flushleft}

\vspace*{1cm}
text the content soon ages--some less gracefully than others. It is becoming more common for Web designers to help readers by attaching a ''last revised on date'' indi-cator to their pages. In the absence of such an indicator or as an accuracy check, most popular browsers provide a way to display a limited set of information about the page (for example, right click and select ''page properties'' in Internet Explorer). This infor-mation includes the author's name and the data the page was last modified. Noting the latest date of any references quoted in the content can also check currency of academic publications. Obviously the more current the document the better, but old pages, like old wine, often have historical, if not legacy value!\\

\vspace*{0.5cm}

\large{
\textbf{Readability}
}

\hspace*{0.7cm} Readability of Net-based literature can be improved in a number of ways from what is available in paper format. First, the text of much Net-based content can be modified by the reader's program to suit viewing preferences. The font can be enlarged for those with visual impairments, or compressed for those wishing to rapidly scan the content. The text can also be imported into word processors or text editors for further elabora-tion such as adding color or emphasis, changing fonts, and printing onto paper. Such manipulation would constitute copyright violation if the document was further dis-tributed but is permitted for single, personal use by the e-researcher.\\

\hspace*{0.7cm} Net-based content also provides opportunity to raise the concept readability to new levels, as the capacity for publication in multimedia formats becomes viable. Sound recordings, animations, video clips, and sophisticated computer models and simulations can be included in either peer-reviewed or popular research literature. As acceptance of e-research grows, the inclusion of such multimedia components of oth-ers' work in research reports will likely become commonplace. It should be noted, however, that care must be taken to insure the authors' own copyright for all media distributed in their presentations and/or publications (seen http://fairuse.stanford.edu/for guidelines on fair use in the United States and http://www.uottawa.ca/library/carl/projects/copyright/c-r-e.htm for a discussion of digital copyright issues from a Cana-dina perspective).\\

\vspace*{0.5cm}

\large{
\textbf{Relevance}\\
}

\hspace*{0.7cm} The quest for relevance of literature review content seems little changed in an e-research context from earlier days. The increased capacity and efficiency of search engines (see Chapter 2) will help e-researchers reduce the time it takes to find and review potential content for inclusion in their literature review. However, the task of assessing relevance and veracity may actually increase in scope and difficulty for e-researchers as they are presented with a constantly increasing body of accessible information. Thus, increased skill and competency is required of the e-researcher. This higher standard for quality and relevance of the literature review is one of the many ways e-research promises to improve the efficiency and efficacy of the research process.\\

\end{document} 